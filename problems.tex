% For Science Olympics on Feb. 5, 2019

\documentclass[11pt]{article}
\usepackage[margin=1in]{geometry}
\usepackage[utf8]{inputenc}

\newcommand{\problem}[2]{\textbf{\Large Problem #1: #2} \vspace{0.4em}}
\newcommand{\heading}[1]{\vspace{0.6em} \textbf{#1}}

\setlength{\parindent}{0pt} % no paragraph indent
\setlength{\parskip}{0.6em}


\title{2019 Centennial CVI Science Olympics}
\author{Computer Science Event}
\date{February 5, 2019}


\begin{document}

\maketitle

\section{Rules}

Probably just use the Google Docs version (and omit this page). However, if there is math to display in the rules, this could be a good tool.

\pagebreak


% ========== PROBLEM 1 ==========

\problem{1}{Cute Codes}


\heading{Problem Description}

It is a well-known fact that cute puppies increase programming productivity. Therefore, the most talented programmers are always on the lookout for the cutest puppies.

Bob and his group of computer science enthusiasts are going puppy shopping, and thus want to find out whether each puppy meets the ``cuteness" criteria. Like any regular puppy shopper would, they have decided to label each puppy with a four-digit ``cute code" in order to determine its cuteness.

They consider a puppy cute if and only if:
\begin{itemize}
    \item the first digit is at least 5; and
    \item the sum of the digits is less then 18.
\end{itemize}

Can you tell Bob and his group of computer scientists if each puppy is cute enough?

\textit{Please note that cute codes of} artificial \textit{puppies will be used in the testing of your program. No actual puppies will have their feelings hurt.}


\heading{Input Specification}

The input will consist of ten lines, each containing a puppy's four-digit code.


\heading{Output Specification}

For each puppy, if it meets the cuteness criteria, output \verb|So cute!|. Otherwise, output \verb|Sorry|.


\heading{Sample Input}
\vspace{-\topsep}
\begin{verbatim}
5212
0123
7890
\end{verbatim}

\vspace{-\topsep}
\heading{Output for Sample Input}
\vspace{-\topsep}
\begin{verbatim}
So cute!
Sorry
Sorry
\end{verbatim}

\vspace{-\topsep}
\heading{Explanation for Output for Sample Input}

The first code \verb|5212| meets both of the cute criteria. However, the first digit of \verb|0123| is too small, and \verb|7890| has a digit sum of 24, which is too big.

Note: the sample input shown only contains three cases, but the real data files will contain ten.


\pagebreak



% ========== PROBLEM 2 ==========

\problem{2}{Overflowing}


\heading{Problem Description}

Stack Overflow is a frequently-visited forum by professional programmers, as it fulfills all of their programming needs. As with many other forums, features include asking and answering questions and, of course, upvoting answers that one finds helpful. Bob, being the enthusiastic programmer he is, has posted answers to $N$ questions. However, not all of his answers are of the best quality: sometimes, other users will choose to give negative upvotes, or \textit{downvotes}. Therefore, the score of an answer can also be negative.

Bob is competing in a contest sponsored by Stack Overflow that rewards a T-shirt for the user with the highest total score, and a pair of socks for the person with the lowest total score. A user must choose $K$ of their posts to submit to the contest, and the user’s score is calculated as the sum of the scores of the selected posts. 

Bob doesn't know whether to aim for the socks or for the T-shirt! Therefore, he is asking your group of programmers for help. To judge his chances of winning, he wants to know: what is the lowest and highest possible score he can attain in the contest?

\heading{Input Specification}

The input will consist of ten test cases.

Each test case contains two lines. The first line contains two space-separated integers, $N$ ($0 \le N \le 10^5$), the number of answers Bob has posted, and $K$ ($0 \le K \le N$), the number of posts Bob can select for the contest. The second line contains $N$ space-separated integers each between $-1000$ and 1000 inclusive: the scores that Bob's $N$ answers have received.

\heading{Output Specification}

For each test case, output two space-separated integers on one line: the minimum and maximum possible score Bob can get by selecting up to $K$ of his $N$ answers.


\heading{Sample Input}
\vspace{-\topsep}
\begin{verbatim}
5 3
9 3 8 8 9
4 2
-7 3 -2 -1
4 4
-100 -100 -100 -100
\end{verbatim}

\vspace{-\topsep}
\heading{Output for Sample Input}
\vspace{-\topsep}
\begin{verbatim}
19 26
-9 2
-400 -400
\end{verbatim}

Note: the sample input shown only contains three cases, but the real data files will contain ten.

% \vspace{-\topsep}
% \heading{Explanation for Output for Sample Input}

\pagebreak



% ========== PROBLEM 3 ==========

\problem{3}{Stargazing}


\heading{Problem Description}

After each long, grueling day of programming, Bob sits down in a rocking chair in his backyard to gaze at the stars. Bob lives in a rural area, so there is no light pollution obscuring his view. Also, it is the middle of summer, so it is just the right temperature to enjoy a good session of stargazing.

Every night, as Bob sips his lemonade, he wishes he had someone with him to enjoy the stars. However, as he feels lonely, his mind wanders off...

As he looks at the stars, he notices that they are not all uniformly distributed across the sky. Remembering the most recent book he read (from two years ago), he defines the \textit{fault} of the stars to be the square of the minimum distance between any two distinct stars.

More formally, the night sky can be represented as a 2D coordinate plane containing $N$ points. If the two stars with minimal distance between them are located at $(x_1, y_1)$ and $(x_2, y_2)$, the fault of the stars is defined to be $(x_1 - x_2)^2 + (y_1 - y_2)^2$.

Given ten arrangements of stars, can you help Bob determine the fault of each arrangement?


\heading{Input Specification}

The input will consist of ten test cases, each corresponding to a night of stargazing.

Each case begins with the value $N$ ($2 \le N \le 10^3$), the number of stars in the night sky. This is followed by $N$ pairs of integers $x, y$ ($-10^4 \le x, y \le 10^4$), the coordinates of a star.

\heading{Output Specification}

For each test case, output one integer on its own line. The answer will always be an integer.


\heading{Sample Input}
\vspace{-\topsep}
\begin{verbatim}
2
0 0
2 0
3
0 1
4 1
2 4
\end{verbatim}

\vspace{-\topsep}
\heading{Output for Sample Input}
\vspace{-\topsep}
\begin{verbatim}
4
13
\end{verbatim}

Note: the sample input shown only contains two cases, but the real data files will contain ten.

% \vspace{-\topsep}
% \heading{Explanation for Output for Sample Input}

\pagebreak



% ========== PROBLEM 4 ==========

\problem{4}{Final Frontier}


\heading{Problem Description}

At long last, Bob has reached the final frontier (and the most dreaded) of his school courses: his gym class. The incredibly bulky gym teacher, Mr. Yava, enjoys torturing his students by making them run laps around the perfectly-circular gymnasium. Every day, while the class is running, Mr. Yava likes to play a fun game called ``Mr. Yava tells you what to do".

Mr. Yava will successively call out $N$ degree values between 0 and 360, inclusive, each interpreted as a value in degrees. Each number yelled specifies the number of degrees Bob must run along the outside of the gymnasium. Bob must follow all of these instructions in the order they are given.

However, there's one catch: Mr. Yava does not specify the direction in which Bob must run. Therefore, for each instruction, Bob can choose to run either clockwise or counterclockwise.

Bob and his classmates begin the workout at the $0^{\circ}$ point of the gym, and Mr. Yava always watches his class standing at the $180^{\circ}$ point of the gym and asks the class to gather in front of him after the workout. Bob, being an non-athletic computer science enthusiast, wants to do as little running as possible, and wants to end exactly where his teacher stands.

Bob is too tired from running around, so he asks your group of programmers for help. How many ways are there for him to reach his teacher (at \textit{exactly} the $180^{\circ}$ point) after performing all $N$ instructions?


\heading{Input Specification}

The input will consist of ten test cases, each corresponding to one day of Bob's gym class.

The first line of each case contains the integer $N$ ($1 \le N \le 30)$, the number of instructions yelled by Mr. Yava. The second and last line of the case contains $N$ space-separated integers, each between 0 and 360 inclusive: the instructions given by Mr. Yava in order.

For 50\% of the available marks, $N \le 15$.


\heading{Output Specification}

Output one line for each test case: the number of ways to reach Mr. Yava at $180^{\circ}$.


\heading{Sample Input}
\vspace{-\topsep}
\begin{verbatim}
3
10 20 150
2
180 180
\end{verbatim}

\vspace{-\topsep}
\heading{Output for Sample Input}
\vspace{-\topsep}
\begin{verbatim}
2
0
\end{verbatim}

Note: the sample input shown only contains two cases, but the real data files will contain ten.

% \vspace{-\topsep}
% \heading{Explanation for Output for Sample Input}


\end{document}
